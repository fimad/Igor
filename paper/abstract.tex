\abstract{

    This paper presents the design and implementation of a metamorphic engine
        intended to be capable of evading current commercial classification
        methods as well as several promising techniques from current
        anti-malware research.  The resulting system is able to achieve high
        levels of variation among the instances generated while at the same time
        creating programs that share key statistical properties with benign
        programs. The engine achieves this by re-framing malware obfuscation as
        a combined compiler and search problem and is heavily influence by
        developments in the field of return-oriented programming.  Given a
        semantic blue print of a program the engine pieces together blocks of
        random bytes to create the final executable. By controlling the
        distribution of the random bytes used to generate the executable, the
        engine is able to influence statistical properties of the resulting
        executables. To achieve the high degree of variation necessary to evade
        commercial detection techniques, the process of piecing together and
        choosing random bytes is performed nondeterministically.  

}
