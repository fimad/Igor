\chapter{Conclusion}

    Commercial antivirus software has historically relied on static techniques
    to detect instances of malware in the wild. Typical implementations rely on
    byte signatures to detect malicious programs. An active area of research is
    creating robust detection techniques capable of coping with metamorphic
    malware which is able to obfuscate its byte signature. An emerging static
    technique that has shown academic success uses knowledge of the structure of
    benign programs to classify malware. A prominent example of this is opcode
    frequency analysis \cite{chisquared}.

    The metamorphic system presented in this paper is shown to be capable of
    producing programs that exhibit a high degree of variation in their byte
    signatures while at the same time maintaining an opcode frequency
    distribution similar to those found in benign Windows programs. It performs
    well compared to state of the art gadget-based metamorphic engines in terms
    of byte signature variance and out performs the state of the art in respect
    to conforming to benign opcode distributions.

    Further, this paper has also provided empirical support for the claim made
    in \cite{franken} that suggests that when nondeterministically stitching
    together a program out of gadgets, the resulting program's overall opcode
    distribution can be controlled by altering the distribution of bytes that
    the gadgets are sampled from.

    By achieving high levels byte signature variance and opcode frequency
    conformity the described engine demonstrates that it is currently feasible
    for malware authors to evade both the current algorithms and emerging static
    detection techniques. 
    %This high degree of obfuscation that state of the art metamorphic systems
    %are able to produce gives weight to the argument that if vendors wish to
    %keep up with evolving metamorphic techniques the industry must move from a
    %statically oriented approach to one that incorporates behavioural analysis
    %as well.

    \section{Future Work}

    The system described in this paper is a proof of concept and as such
    contains many areas that would benefit from further investigation.

    \subsection{Improvements In Modelling Benign Programs}

    The byte source proposed by this paper uses a unigram byte model to describe
    byte distributions and is a trivial improvement over the naive approach of
    sampling gadgets from a uniform distribution. It would be beneficial to
    examine the performance of the engine under more complicated models such as
    general $n$-gram models, and models based on opcodes frequencies rather than
    byte frequencies.

    There is also the potential of incorporating the model into the code
    generation algorithm by choosing gadgets based on their probability instead
    of the current approach of uniformly randomly choosing them.

    \subsection{Code Generation}

    There are several pruning techniques used in the engine that limit the
    variation in the programs that can be generated. Some of these can be
    accounted for by incorporating existing techniques, such as intelligent
    register allocation \cite{register-coloring}.

    Others require more specialized algorithms. This paper presents a novel
    approach for generating constants not found in the gadget library that
    achieves reasonable performance at the cost of variability. A more general
    approach that does not sacrifice variability would be desirable in a mature
    system.

    \subsection{Abstract Evaluator}

    Building a large library of gadgets that covers a wide range of operations
    is necessary to create complex programs. The current implementation handles
    only $19$ of the over $500$ unique $x86$ opcodes. It would be
    interesting to attempt to correlate the performance of the system to the
    number of supported opcodes.

    Similarly, in order to take full advantage of the supported opcodes, the
    evaluator would need to handle semantically equivalent by syntacticly
    distinct expressions, e.g. $(1+1)*eax = 2*eax = (eax+eax)$.

%    \subsection{Analysis}
%
%    Given a more mature implementation, encrypt existing malware
%    use real world software in classification  instead of toy metrics
