\chapter{Introduction}

    %Write everything down broadly in a single paragraph!
    Modern malware relies on sophisticated obfuscation techniques to evade
    detection by antivirus software. In order to build better classifiers, it is
    useful to understand the limits and capabilities of these obfuscation
    techniques. This paper presents a new system for program obfuscation with
    two primary design goals. The first is that the newly generated executables
    appear dissimilar to each other. The second is that the programs appear
    statistically similar to unobfuscated programs, thereby disguising the fact
    that any obfuscation was performed.

%    what is malware

%    The term malware describes a broad class of malicious programs that are
%    intended to gather sensitive information, disrupt normal operation or
%    otherwise subvert control of an non-consenting host machine. A survey
%    performed by Microsoft in 2013, shows that approximately $1\%$ of all
%    Windows XP machine surveyed were infected with some form of malware
%    \cite{infect_rates}. To protect users from the abundance of malevolent
%    programs in the wild, an entire industry has developed centered around
%    detecting and preventing the spread of malware.

%    how does it historically work

    Historically malware detection has relied on various forms of static
    analysis, or techniques that attempt to identify malware prior to its
    execution.  Static analysis has been favored because it has the desirable
    property that it is relatively quick, and when it works, it is able to stop
    malware before it is allowed to interact with the host. The classic example
    of a static method of detection, and one that still plays a major role in
    commercial antivirus software solutions, is byte signature matching.  Byte
    signature matching classifies malware by checking if the file under
    consideration matches the byte signature of any known instances of malware.

%    how does malware work currently

    Byte signature matching has the major drawback that it requires constant
    maintenance of a database of signatures in order to be effective. This means
    that newly minted malware enjoys a brief period where it is able to avoid
    detection before the databases have been updated to include its signature.
    In an attempt to thwart detection, malware authors have created systems that
    target this weakness. Modern malware typically uses various obfuscation
    schemes to scramble its byte signature every time a new host is infected.
    Because each infection has a unique byte signature it is not possible to
    create a universal byte signature that can classify each instance of a
    specific piece of obfuscating malware. This type of malware poses a serious
    problem for antivirus vendors.

%    what's being done to stop it

    The classification of obfuscating, or metamorphic, malware is currently an
    open research problem. There are several emerging methods of detecting
    metamorphic malware that have been shown to perform well in lab settings.
    One category of techniques uses available knowledge of the host environment
    in order to statically classify obfuscated malware. In practice, the systems
    used to disguise malware are very good at creating new instances that appear
    very different from each other, however, as a side effect these instances
    also appear very different from normal benign programs. As a result,
    successful detection systems have been created that simply compare the
    opcode frequencies of a program under consideration to those of ``normal''
    programs \cite{histogram}.

    In addition to the historically favored static methods, dynamic techniques
    that observe the execution of an instance of malware have shown promising
    results as well. These systems make the observation that while the byte
    level appearance of an obfuscated malware may change, how it interacts with
    the host largely remains the same. Examples of such methods include
    fingerprinting the behaviour of a program by the recording the sequence of
    syscalls made during execution.

%    why that won't work
    
    This paper presents a new type of metamorphic system that is capable of
    producing semantically equivalent programs which exhibit a high degree of
    variation in their byte signatures while at the same time remaining
    statistically similar to benign programs. This new engine achieves this by
    reframing malware obfuscation as a combined compiler and search problem.
    Given a semantic blue print of a program the engine pieces together blocks
    of random bytes to create a working executable. By controlling the
    distribution of the random bytes used to generate the executable, the engine
    is able to influence statistical properties of the resulting executables. To
    achieve the high degree of variation necessary to evade byte signature
    detection, the process of piecing together and choosing random bytes is
    performed nondeterministically.
