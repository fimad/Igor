\chapter{Introduction}

    The term malware describes a broad class of programs that perform malicious
    activities without the consent of the user. Examples of such activities
    include hosting illicit content, displaying ads, sending spam, and spying on
    user activity. 
%    Programs in this class include computer viruses and worms
%    which infect machines without user consent, and Trojans which claim to
%    perform one a benign task while actually or additionally performing
%    nefarious tasks.
    To protect users from these types of malicious programs, an
    entire industry has developed centered around detecting and analyzing
    malware.

    Malware research is often framed as a back and forth battle between
    antivirus vendors and malware authors. Vendors create systems that detect
    and prevent the execution of malware, and in response malware authors create
    more sophisticated systems that are able to evade detection. In response to
    the new types of malware, vendors research and deploy new detection
    techniques and the cycle repeats ad infinitum. 

    Currently, commercial antivirus software largely relies on what is known as
    static detection techniques, the most common being byte signature matching.
    In this system, each vendor maintains a database of signatures for every
    piece of malware that has ever been identified. Classification is done by
    comparing the bytes in a sample file against the signatures in the database,
    if there is a match then the sample is classified as malicious.

    To thwart byte signature based techniques, malware authors create malware
    that is capable of altering it's appearance every time it infects a new
    host. The methods used to perform this obfuscation have become more and more
    complex over time, ranging from simple encryption schemes, to executing the
    malware logic in custom virtual machines. An unintended result of
    obfuscation is that the byte level appearance of malware has diverged from
    that of benign programs. This observation spawned a current branch of
    research in malware detection which has shown that obfuscated malware can be
    detected by comparing the malware's byte and opcode distributions to those
    of benign programs.

    As malware authors develop increasingly complex methods of obfuscating their
    programs, antivirus vendors must in turn create sophisticated methods of
    detection. Unfortunately for vendors, there is a theoretical bound on the
    efficiency of a robust static classification system\cite{npcomplete}. To
    demonstrate the need for the inclusion of dynamic detection techniques, this
    paper presents a metamorphic system capable of producing semantically
    equivalent programs which exhibit a high degree of variation in their byte
    signatures while at the same time remaining statistically similar to benign
    programs.

    This engine achieves this by re-framing malware obfuscation as a combined
    compiler and search problem. Given a semantic blue print of a program the
    engine pieces together blocks of random bytes to create a working
    executable. By controlling the distribution of the random bytes used to
    generate the executable, the engine is able to influence statistical
    properties of the resulting executables. To achieve the high degree of
    variation necessary to evade byte signature detection, the process of
    piecing together and choosing random bytes is performed
    nondeterministically.

