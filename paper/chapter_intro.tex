\chapter{Introduction}

%    what is malware

    The term malware describes a broad class of malicious programs that are
    intended to gather sensitive information, disrupt normal operation or
    otherwise subvert control of an non-consenting host machine. A survey
    performed by Microsoft in 2013, shows that approximately $1\%$ of all
    Windows XP machine surveyed were infected with some form of malware
    \cite{infect_rates}. To protect users from the abundance of malevolent
    programs in the wild, an entire industry has developed centered around
    detecting and preventing the spread of malware.

%    how does it historically work

    Historically malware detection has relied on various forms static analysis,
    or techniques that attempt to identify malware prior to its execution.
    Static analysis has been favored because it has the desirable properties
    that it is relatively quick, and when it works, it is able to stop malware
    before it is allowed to interact with the host. The classic example of a
    static method of detection, and one that still plays a major role in
    commercial antivirus software solutions, is byte signature matching. Byte
    signature matching classifies malware by checking if then file under
    consideration matches the byte signature of any known instances of malware.

%    how does malware work currently

    Byte signature matching has the major drawback that it requires constant
    maintenance of a database of signatures in order to be effective. This means
    that newly minted malware enjoys a brief period where it is able to avoid
    detection before the databases have been updated to include its signature.
    In an attempt to thwart detection, malware authors have created systems that
    target this weakness. Modern malware typically uses various obfuscation
    schemes to scramble its byte signature every time a new host is infected.
    Because each infection has a unique byte signature it is not possible to
    create a universal byte signature that can classify each instances of a
    specific piece of obfuscating malware. This type of malware poses a serious
    problem for antivirus vendors.

%    what's being done to stop it

    The classification of obfuscating, or metamorphic, malware is currently an
    open research problem. There are several emerging methods of detecting
    metamorphic malware that have been shown to perform well in lab settings.
    One category of techniques uses broader knowledge of the host environment in
    order to statically classify obfuscated malware. In practice, the systems
    used to disguise malware are very good at creating new instances that appear
    very different from each other, however, as a side effect these instances
    also appear very different from normal benign programs. As a result,
    successful detection systems have been created that simply compare the
    opcode frequencies of a program under consideration to those of ``normal''
    programs.

    In addition to the historically favored static methods, dynamic techniques
    that observe the execution of an instance of malware have shown promising
    results as well. These systems make the observation that while the byte
    level appearance of an obfuscated malware may change, how it interacts with
    the host largely remains the same. Examples of such methods include
    fingerprinting the behaviour of a program by the recording the sequence of
    syscalls made during execution.

%    why that won't work
    
    This paper is intended to demonstrate the weakness of static techniques,
    both current and emerging, and make a case for the inclusion of dynamic
    detection schemes. A new type of metamorphic system is presented that is
    capable of producing semantically equivalent programs which exhibit a high
    degree of variation in their byte signatures while at the same time
    remaining statistically similar to benign programs.

    This new engine achieves this by re-framing malware obfuscation as a
    combined compiler and search problem. Given a semantic blue print of a
    program the engine pieces together blocks of random bytes to create a
    working executable. By controlling the distribution of the random bytes used
    to generate the executable, the engine is able to influence statistical
    properties of the resulting executables. To achieve the high degree of
    variation necessary to evade byte signature detection, the process of
    piecing together and choosing random bytes is performed
    nondeterministically.
     

%    activities without the consent of the user. Examples of such activities
%    include hosting illicit content, displaying ads, sending spam, and spying on
%    user activity. 
%%    Programs in this class include computer viruses and worms
%%    which infect machines without user consent, and Trojans which claim to
%%    perform one a benign task while actually or additionally performing
%%    nefarious tasks.
%    To protect users from these types of malicious programs, an
%    entire industry has developed centered around detecting and analyzing
%    malware.
%
%    Malware research is often framed as a back and forth battle between
%    antivirus vendors and malware authors. Vendors create systems that detect
%    and prevent the execution of malware, and in response malware authors create
%    more sophisticated systems that are able to evade detection. In response to
%    the new types of malware, vendors research and deploy new detection
%    techniques and the cycle repeats ad infinitum. 
%
%    Currently, commercial antivirus software largely relies on what is known as
%    static detection techniques, the most common being byte signature matching.
%    In this system, each vendor maintains a database of signatures for every
%    piece of malware that has ever been identified. Classification is done by
%    comparing the bytes in a sample file against the signatures in the database,
%    if there is a match then the sample is classified as malicious.
%
%    To thwart byte signature based techniques, malware authors create malware
%    that is capable of altering it's appearance every time it infects a new
%    host. The methods used to perform this obfuscation have become more and more
%    complex over time, ranging from simple encryption schemes, to executing the
%    malware logic in custom virtual machines. An unintended result of
%    obfuscation is that the byte level appearance of malware has diverged from
%    that of benign programs. This observation spawned a current branch of
%    research in malware detection which has shown that obfuscated malware can be
%    detected by comparing the malware's byte and opcode distributions to those
%    of benign programs.
%
%    As malware authors develop increasingly complex methods of obfuscating their
%    programs, antivirus vendors must in turn create sophisticated methods of
%    detection. Unfortunately for vendors, there is a theoretical bound on the
%    efficiency of a robust static classification system\cite{npcomplete}. To
%    demonstrate the need for the inclusion of dynamic detection techniques, this
%    paper presents a metamorphic system capable of producing semantically
%    equivalent programs which exhibit a high degree of variation in their byte
%    signatures while at the same time remaining statistically similar to benign
%    programs.
%
%    This engine achieves this by re-framing malware obfuscation as a combined
%    compiler and search problem. Given a semantic blue print of a program the
%    engine pieces together blocks of random bytes to create a working
%    executable. By controlling the distribution of the random bytes used to
%    generate the executable, the engine is able to influence statistical
%    properties of the resulting executables. To achieve the high degree of
%    variation necessary to evade byte signature detection, the process of
%    piecing together and choosing random bytes is performed
%    nondeterministically.

