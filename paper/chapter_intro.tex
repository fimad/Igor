\chapter{Introduction}

    Malware and security research is often framed as a back and forth exchange
    between ``white hat'' researchers and antivirus vendors on one side and
    ``black hat'' malware authors on the other. Malware authors write malicious
    programs which infect machines of unsuspecting users and perform various
    nefarious acts such as sending spam, or spying on the user's activity. In
    response an entire antimalware industry has sprung up centered around
    protecting users from malicious programs.

    Current commercial antivirus software largely relies on static detection
    techniques to classify malware. The most common static technique is
    byte signature matching where each vendor maintains a database of signatures
    for every piece of malware that has ever been identified. Classification is
    done by comparing the bytes in a sample file against the signatures in the
    database, if there is a match then the sample is classified as malicious.

    In response to this technique, malware authors created malware that was
    capable of altering it's appearance itself each time it infecting a new
    host. The methods used to perform this obfuscation have become more and more
    complex over time, ranging from simple encryption of the malware, to
    executing the malware in custom virtual machines. As obfuscation techniques
    became more advanced the appearance of obfuscated malware diverged from
    benign programs. Which led to a current branch of research in malware
    detection which has shown that obfuscated malware can be detected by
    comparing the byte and opcode distributions to benign programs.

    As malware authors develop increasingly complex methods of obfuscating their
    programs antivirus vendors must in turn create sophisticated methods of
    detection. Unfortunately for vendors, there is a theoretical bound on the
    efficiency of a robust static classification system\cite{npcomplete}. To
    demonstrate the need for the inclusion of dynamic detection techniques, this
    paper presents a metamorphic system capable of producing semantically
    equivalent programs which exhibit a high degree of variation in their
    byte signatures while at the same time remaining statistically similar to
    benign programs.

    This engine is able to achieve this by re-framing malware obfuscation as a
    combined compiler and search problem. The engine takes a semantic blue print
    of a program and then pieces together blocks of random bytes to create a
    working executable. By controlling the distribution of the random bytes used
    to generate the executable, the engine is able to influence statistical
    properties of the resulting executables. The engine is able to achieve a
    high degree of variation in byte signatures by making the process
    nondeterministic which causes each of the programs generated are composed of
    different blocks bytes.

