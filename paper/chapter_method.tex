\chapter{Method}

    This section describes the theoretical underpinnings and the implementation
    of a gadget-based metamorphic engine. The engine maintains a large library
    that maps from discrete operations to bytes that perform those operations
    when executed. Leveraging this library and a semantic blueprint of a desired
    program, the engine generates an executable by nondeterministically piecing
    together bytes from the library in a way that fulfils the semantics of the
    program.

    \section{Gadgets}
    
    The idea of stitching together programs out of existing building blocks
    originated in the field of return-oriented programming \cite{rop_geo}.
    Return-oriented programming chains together small snippets of instructions
    ending in return statements by manipulating the return address on the stack.
    These snippets of instructions are called gadgets in the return-oriented
    programming literature and each gadget typically performs a single simple
    operation.

    The traditional formalization and use of gadgets has been heavily influenced
    by the constraints present in return-oriented programming. Many of these
    imposed constraints are not present in the context of a metamorphic engine.
    For instance, the requirement that a gadget end in a return statement is
    necessary in return-oriented programming because that is the mechanism
    through which gadgets can be composed. However, given that this engine
    directly generate the text section of an executable this requirement is no
    longer necessary.

    Therefore, it is useful to re-formalize gadgets in the context where they
    are free of the constraints imposed by return-oriented programming. For the
    purposes of this engine, gadgets are defined as a byte string and a
    corresponding semantics given as a symbolic description of the effect on the
    execution state the byte string would have upon execution.

    To simplify the process of reasoning about a large collection of gadgets, a
    given byte string is restricted to matching several of a collection of
    predefined gadget types. A Turing complete set of gadget types is given in
    \cite{franken}, and serves as the basis for the gadget types in this
    implementation. Each type of gadget is parameterized over a list of hardware
    locations\footnote{Memory addresses and registers}. A complete listing of
    gadgets and their definitions is given in table~\ref{tab:method-gadgets}.

    \begin{table}
        \centering
        \begin{tabular}{llL}
            \hline
            Gadget Type & Parameters & Semantics \tabularnewline
            \hline
            NoOp & --- & --- \tabularnewline
            LoadReg & $out,in$ & $out \leftarrow in$ \tabularnewline
            LoadConst & $out,value$ & $out \leftarrow value$ \tabularnewline
            LoadMemReg & $outReg,in,value$ & $outReg \leftarrow [in + value]$ \tabularnewline
            StoreMemReg & $outReg,value,inReg$ & $[outReg+value] \leftarrow inReg$ \tabularnewline
            Plus & $out,in_1,in_2$ & $out \leftarrow in_1 + in_2$ \tabularnewline
            Minus & $out,in_1,in_2$ & $out \leftarrow in_1 - in_2$ \tabularnewline
            Times & $out,in_1,in_2$ & $out \leftarrow in_1 \times in_2$ \tabularnewline
            Xor & $out,in_1,in_2$ & $out \leftarrow in_1 \oplus in_2$ \tabularnewline
            RightShift & $reg,value$ & $reg \leftarrow reg >> value$ \tabularnewline
            Compare & $in_1,in_2$ & $eflag \leftarrow compare(in_1, in_2)$ \tabularnewline
            Jump & $offset,reason$ & 
            \[ eip \leftarrow
                \begin{cases}
                    eip + offset & \text{if } eflag \& reason\\ 
                    eip & otherwise 
                \end{cases}
            \] \tabularnewline
            \hline
        \end{tabular}
        \caption[List of gadgets.]{Enumeration of gadgets and their semantic definitions.}
        \label{tab:method-gadgets}
    \end{table}

    Along with the core semantics presented in table~\ref{tab:method-gadgets},
    each byte string also has an associated ``clobber set''. This set contains
    the set of hardware locations that the gadget overwrites in addition to
    performing it's core semantics. Allowing gadgets to have limited
    side-effects describable by this clobber set allows for junk
    instructions to be present in the byte strings for any given gadget type.
    This is a desirable feature for a metamorphic engine because it leads to
    greater variety in the byte instances for each gadget type and through that
    more variety in the generated programs. Concrete examples of gadgets along
    with their clobber sets are shown in table~\ref{tab:method-clobber}.

    \begin{table}
        \centering
        \makebox[\textwidth][c]{
            \begin{tabular}{LLLL}
                \hline
                Bytes & Instructions & Gadget Type & Clobber Set \tabularnewline
                \hline\\
                \texttt{01CB A100} \\
                \texttt{0000 00} 
                &
                add ebx, ecx \\
                mov eax, \$10
                &
                Plus($ebx$,$ebx$,$ecx$)
                &
                $eax$
                \tabularnewline \\ \hline \\
                \texttt{330D 0000} \\
                \texttt{0000 8998} \\
                \texttt{0000 0000} \\
                \texttt{29CA}
                &
                xor ecx, \$42 \\
                mov [eax+\$10], ebx \\
                sub edx, ecx
                &
                StoreMemReg($eax$, $10$, $ebx$)
                &
                $ecx$, $edx$
                \tabularnewline \\
                \hline
            \end{tabular}
        }
        \caption[Concrete examples of gadgets.]
        {Concrete examples of gadgets along with their associated clobber set}
        \label{tab:method-clobber}
    \end{table}

    \section{Semantic Blueprints}

    Gadgets as they are used in this engine occupy a role similar to assembly
    instructions in traditional code generation. They perform simple operations
    and are parameterized over hardware locations. Just as writing a program in
    pure assembly would require manually maintaining such things as stack frames
    and mappings of registers to variables, so too would attempting to write a
    program in pure gadgets. 

    Therefore to provide a more convenient programming experience it is
    necessary to be able to express program semantics using a higher level
    construct. The solution used in this implementation is ``abstract
    statements''. Abstract statements, or simply statements, are similar to
    gadgets in that the scope of operations they perform but abstract away the
    hardware locations into variables. This divorces the program semantics from
    the specific hardware that it will be running on, and opens the door for
    cross-platform compilation.

    \begin{table}
        \centering
        \begin{tabular}{llL}
            \hline
            Gadget Type & Parameters & Gadget \tabularnewline
            \hline
            noop & --- & NoOp \tabularnewline
            move & $out,in$ & $out \leftarrow in$ \tabularnewline
            add & $out,in_1,in_2$ & $out \leftarrow in_1 + in_2$ \tabularnewline
            sub & $out,in_1,in_2$ & $out \leftarrow in_1 - in_2$ \tabularnewline
            mul & $out,in_1,in_2$ & $out \leftarrow in_1 \times in_2$ \tabularnewline
            xor & $out,in_1,in_2$ & $out \leftarrow in_1 \oplus in_2$ \tabularnewline
            jump & $label,reason$ & 
            \[ eip \leftarrow
                \begin{cases}
                    eip + offset(label) & \text{if } eflag \& reason\\ 
                    eip & otherwise 
                \end{cases}
            \] \tabularnewline
            \hline
        \end{tabular}
        \caption[List of abstract statements.]
        {Enumeration of abstract statements and their semantic definitions. Note
        that parameters here are either variables or constants and that each
        statement has a clear correspondence with a gadget in
        \ref{tab:method-gadgets}}
        \label{tab:method-statements}
    \end{table}

    \section{Gadget Discovery}

    In order to create programs out of these gadgets, it is first necessary to
    actually find byte strings that match the described gadget types. Therefore
    the first stage of the metamorphic engine is to pre-generate a large library
    that maps from parameterized gadgets to byte strings that implement them.

    \subsection{Byte Source}

    The first thing required to begin matching byte strings to gadgets and
    inserting them into the library is to determine where the bytes are going to
    come from. The method proposed by \cite{franken} is to scan the host
    computer and read bytes from existing executables. This approach has the
    desirable property that the bytes are being sampled from benign programs
    which causes them to exhibit a similar frequency distribution.  There is a
    fatal problem with this approach however, and that is that this approach
    requires interaction with the host environment in a way that is easily
    recognizable by behavioural detection systems due to it's uncommonly large
    number of disk seeks \cite{anon_evade}.

    A naive alternative to this approach would be to randomly generate bytes
    from a uniform distribution. This solution has several beneficial
    properties, namely there is no longer any interaction with the host
    environment at all, and there is no longer a limited number of bytes that
    may be sampled. Despite these niceties there is a significant downside to
    this approach. The distribution that the bytes are being sampled from no
    longer has any connection to the distribution of bytes found in benign
    programs.

    A natural improvement to the naive approach is to pre-calculate the
    distribution of bytes found in benign programs and randomly sample bytes
    from this calculated distribution. This approach maintains all of the
    desirable benefits of the naive approach while avoiding the shortcomings of
    both the naive approach and the approach proposed in \cite{franken}. In
    addition, there is also the additional benefit that since the distribution
    is calculated in advance, it is no longer tied to the host architecture,
    which opens the door for cross-platform code generation.

    \subsection{Symbolic Evaluation}

    In order to match a given byte string with the various gadget types that it
    may implement, it is necessary to be able to represent the effect on
    hardware that the execution of the byte string will have. Modelling the
    effect on hardware is done by disassembling the byte string and symbolically
    evaluating the corresponding instructions in sequence. The result is a
    series of execution-state forests that represent the progressive effect on
    hardware that executing the byte string would have.

    Each state forest contains a tree for each altered hardware location. In
    turn, each tree in the forest is an expression tree where nodes are
    algebraic operations and leaves are the initial values of hardware
    locations. An example of a state forest is given in
    figure~\ref{fig:method-state}.

    \begin{figure}
        \centering
        \begin{tabular}{ccccccc}
            \hline
            \multicolumn{1}{|c|}{$eax$} &
            \multicolumn{1}{c|}{...} &
            \multicolumn{1}{c|}{$eflag$} &
            \multicolumn{1}{c|}{...} &
            \multicolumn{1}{c|}{0xdeadbeef} &
            \multicolumn{1}{c|}{...} \\
            \hline
            \Tree[.$+$ [.$\times$ $eax$ $ebx$ ] $ecx$ ]
            & & 
            \Tree[.$cmp$ $ecx$ $edx$ ]
            & &
            \Tree[. $edi$ ]
            \\
        \end{tabular}
        \caption{Example of a state forest produced by abstract evaluator.}
        \label{fig:method-state}
    \end{figure}

    The types of algebraic operations that are allowed in the expression trees
    are limited to effects which are unconditional and deterministic. Further
    each tree in the forest is required to correspond to a unique hardware
    location. This eliminates possible ambiguity which would further complicate
    the matching process and allows gadgets to be composed without worrying
    about unforeseen side effects.
    
    \subsection{Gadget Matching}

    Each type of gadget recognized by the engine has a corresponding general
    state tree that describes it's core semantics. Examples of what these state
    trees look like are given in table~\ref{tab:method-general-trees}. The wild
    card nodes are constrained to only match the leaf nodes which correspond to
    the parameters of the gadget types described above.

    It would be theoretically possible to match gadgets on entire subtrees
    instead of just leaf nodes. This would allow gadgets to be discovered that
    performed intricate compound operations, and potentially increase the
    variability of the programs the engine is capable of producing. However this
    flexibility would come at a significant price, which is that the number of
    unique parameterizations of any given gadget type would be unbounded as each
    parameter could be any valid expression tree.

    \begin{figure}
        \centering
        \makebox[\textwidth][c]{
            \begin{tabular}{ccc}
                \hline
                Plus & LoadReg & StoreMemReg \\
                \hline
                \\
                    \begin{tabular}[b]{ccc}
                        \hline
                        \multicolumn{1}{|c|}{...} &
                        \multicolumn{1}{c|}{$out$} &
                        \multicolumn{1}{c|}{...} \\
                        \hline
                        & 
                        \Tree[.$+$ $in_1$ $in_2$ ]
                        & \\
                    \end{tabular}
                    &
                    \begin{tabular}[b]{ccc}
                        \hline
                        \multicolumn{1}{|c|}{...} &
                        \multicolumn{1}{c|}{$out$} &
                        \multicolumn{1}{c|}{...} \\
                        \hline
                        & 
                        \Tree[. $inReg$ ]
                        & \\
                    \end{tabular}
                    &
                    \begin{tabular}[b]{ccc}
                        \hline
                        \multicolumn{1}{|c|}{...} &
                        \multicolumn{1}{c|}{$[out+value]$} &
                        \multicolumn{1}{c|}{...} \\
                        \hline
                        & 
                        \Tree[. $inReg$ ]
                        & \\
                    \end{tabular} \\
                    
            \end{tabular}
        }
        \caption{Example of a state forest produced by abstract evaluator.}
        \label{fig:method-general-trees}
    \end{figure}

    Matching a byte string and a corresponding sequence of instructions to
    gadgets types is done by attempting to match each node of the gadget's
    general tree with any tree in the byte string's state forest. The hardware
    locations that correspond to the unmatched trees in the state forest make up
    the clobber set of the matched gadget.

    In order to ensure that the clobber set is able to entirely account for all
    of the side effects of executing the byte string there are restrictions that
    are placed on the unmatched expression trees which must be fulfilled for the
    match to be accepted. Namely, that the expression trees must be ``safe'' to
    execute regardless of the current execution state. To achieve this the
    unmatched trees are only allowed to correspond to writes to registers, and
    are not allowed to include any memory accesses as it is not possible to
    guarantee which locations in memory will be in the resulting program's
    address space.

    \section{Code Generation}

    Generating executables from a semantic blueprint is done by taking a
    representation of a program composed of a collection of methods, each
    described as a sequence of statements, and resolving the statements into
    gadgets and then resolving each gadget into actual bytes.

    In order to translate statements into gadgets it is necessary to first
    assign variables to hardware locations. This requires keeping track of which
    registers and memory locations are free and which have already been assigned
    to variables. This paper uses a naive approach that assigns all local
    variables and method arguments as memory addresses offset from the
    \emph{EBP} register. A fully mature implementation of this engine would use
    a more efficient scheme and there is compiler literature to be drawn on
    \cite{register-coloring}, \cite{register-iterate}, however the naive
    approach is adequate for the purposes of demonstration.

    Once a mapping of local variables to hardware locations has been determined
    it is possible for statements to be translated in to gadgets. Generating the
    bytecode code for each method can then be done by sequentially retrieving
    byte strings from the gadget library for each gadget. 

    \subsection{Variable Mapping \& Statement to Gadget Translation}

    Because gadgets can only be parameterized over hardware locations and not
    local variables, a variable mapping must be generated before statements can
    be translated into gadgets. Once a mapping exists, it is possible to
    directly compile statements into gadgets by substituting gadgets for
    equivalent statements. However while this approach may theoretically work,
    in practice this is highly inefficient.
    
    \begin{figure}
        \centering
        \makebox[\textwidth][c]{
            \begin{tabular}{lcl}
            Statements & & Gadgets \\
            \hline
            move($[ebp+8]$,$10$)
            & &
            LoadConst($[ebp+8]$,$10$) \\

            sub($[ebp+4]$,$[ebp+8]$,$[ebp+16]$)
            & &
            Minus($[ebp+4]$,$[ebp+8]$,$[ebp+16]$) \\

            mul($[ebp+4]$,$[ebp+4]$,$[ebp+4]$)
            & &
            Times($[ebp+4]$,$[ebp+4]$,$[ebp+4]$) \\
            \end{tabular}
        }
        \caption{Example of compiling statements directly to gadgets.}
        \label{fig:method-direct-to-gadget}
    \end{figure}

    Recall that in this implementation, gadgets correspond to short sequences of
    $x86$ instructions typically on the order of 1-4 instructions long. Most
    $x86$ instructions only operate on registers. If we were to naively
    translate statements into a single gadget, the gadget would have to include
    the instructions that load values from memory into registers and then if
    applicable store the value back in memory. The odds of a random sequence of
    instructions cooperating in this manner appearing randomly is very small.
    Therefore to make search a practical method of code generation, it is
    necessary to prune search paths that are unlikely to be fruitful.

    To increase the likelihood that any given gadget will be present in the
    library, statement translation explicitly generates gadgets that load
    variables from memory addresses into temporary registers. An example of a
    possible translation is given below. Note that each gadget is tagged with
    the registers that were bound at the moment of translation. This information
    will be required later when translating gadgets to bytes to ensure that no
    temporary registers are overridden as a side effect of the gadget.

    \begin{figure}
        \centering
        \makebox[\textwidth][c]{
            \begin{tabular}{lcL}
            Statements & Gadgets \tabularnewline
            \hline \\
            move($[ebp+8]$,$10$)
            & &
            {
                LoadConst($eax$,$10$) $[eax]$ \\
                StoreMemReg($ebp$,$8$, $eax$) $[eax]$ \\
            } \tabularnewline

            sub($[ebp+4]$,$[ebp+8]$,$[ebp+16]$)
            & &
            {
                LoadMemReg($eax$, $[ebp+8]$) $[eax]$ \\
                LoadMemReg($ebx$, $[ebp+16]$) $[eax,ebx]$ \\
                Minus($ecx$, $eax$, $ebx$) $[eax,ebx,ecx]$ \\
                StoreMemReg($ebp$,$8$,$ecx$) $[ecx]$ \\
            } \tabularnewline

            mul($[ebp+4]$,$[ebp+4]$,$[ebp+4]$)
            & &
            {
                LoadMemReg($eax$, $[ebp+4]$) $[eax]$ \\
                Times($eax$, $eax$, $eax$) $[eax]$ \\
                StoreMemReg($ebp$,$4$,$eax$) $[eax]$ \\
            } \tabularnewline
            \end{tabular}
        }
        \caption{Example of compiling statements to gadgets with explicit memory
        accesses. The registers in square brackets correspond to registers that
        are not allowed to be clobbered while translating that gadget.}
        \label{fig:method-direct-to-gadget}
    \end{figure}

    \subsection{Gadget to Byte Translation}

    Translating bytes to gadgets is relatively straight forward in most cases
    and can be accomplished by randomly selecting a byte string from the gadget
    library for any encountered gadget.

    The primary constraint is ensuring that the chosen byte string does not have
    any unintended side effects. Recall that effects on hardware that are
    outside the scope of the core semantic of a given gadget are allowed and
    that such effects are encapsulated in each gadget's ``clobber set''. The
    clobber set contains a set of hardware locations that will be in an
    undefined state after execution of the gadget.

    Therefore to ensure that a gadget does not have unintended effects on the
    execution state it is sufficient to require that the gadget's clobber set be
    a subset of the unallocated registers. It is only necessary to consider
    registers because in the creation of the gadget library memory accesses were
    disallowed from being present in the clobber set.

    In the case that there are no byte strings available for a given gadget, the
    algorithm backtracks and tries another translation of the current statement
    using different temporary registers. If every combination of temporary
    registers is tried and a translation is still impossible, the algorithm
    continues backtracking all the way to the top of the search tree to where
    variables are mapped. If backtracking this far is still insufficient the
    engine terminates with the conclusion that the gadget library is too small
    and does not provided adequate coverage. 

    \subsection{Edge Cases}

    The above works well for most types of gadgets but there are several edge
    cases that require further optimizations to make the engine practical. The
    first is the assignment of constants to variables and the second is forward
    jumps.

    \subsubsection{Constants}

    The naive approach to gadget assignment would be to directly translate an
    assignment statement into a corresponding \emph{LoadConst} gadget. However,
    for such a gadget to be found during the construction of the library it is
    necessary that a particular 32 bit value be present in the byte string at
    the correct offset. This is a very unlikely and it would also be impractical
    to store each $2^{32}$ possible \emph{loadConst} gadget in the library. 

    A natural solution is to attempt to construct the equivalent of an
    \emph{LoadConst} gadget out of gadgets that exist in the library. There are
    some caveats with this approach, however, namely that it is reminiscent of
    the subset sum problem\footnote{The subset problem aims to find a subset of
    a set of integer whose sum is equal to a target integer} which is
    NP-complete.

    The following is a novel solution whose runtime is independent of the number
    of bits in the constant being loaded. Loading an $n$-bit constant is done
    using two gadgets. The first loads a constant where the most significant
    bits match the lower $n$ bits of the desired constant. The second gadget
    performs an arithmetic right shift of $32-n$ on this value to generate the
    desired constant.

    The $x86$ shift instruction takes as an operand a $8$ bit number for the
    shift amount. This approach significantly improves the odds of successfully
    compiling an assignment statement to bytes because it relies on the odds of
    an $8$ bit and an $n$ bit number being generated independently rather than a
    specific $32$ bit integer.

    \subsubsection{Jumping Forward}

    The second edge case that requires special handling is forward jumps. The
    problem arises because jump statements are written with relative offsets in
    terms of statements while gadgets have offsets expressed in bytes. This
    means that in order to resolve jump statements to gadgets it is necessary to
    know the byte offset of the statement that the jump is trying to jump to.

    This becomes a problem when dealing with code that attempts to jump forward,
    for example see the example code in figure~\ref{fig:method-forward-jump}.
    In order to know the byte offset of the target statement, the current
    statement must be compiled first which results in a circular dependency.

    %TODO: Forward jump figure
    \begin{figure}
        \centering
        \makebox[\textwidth][c]{
            \begin{tabular}{LcL}
                Statements & & \\
            \hline

            jump(2,$always$)  \\
            sub($[ebp+4]$,$[ebp+4]$,$1$) \\
            add($[ebp+4]$,$[ebp+4]$,$1$)
            & &
%            \parbox{2cm}{
%                \setlength{\unitlength}{.5cm}
%                \begin{picture}(3,3)
%                \thicklines
%                \put(.5,2.3){\line(0,-5){1}}
%                %\put(-2,0){\vector(1,0){1}}
%                \end{picture}
%            }

            \tabularnewline
            \end{tabular}
        }
        \caption{Example of compiling statements to gadgets with explicit memory
        accesses.}
        \label{fig:method-forward-jump}
    \end{figure}

    The proposed method of resolving the circular dependency is to perform
    compilation in two phase. During the first phase non-jump statements are
    translated into bytes as described above. However, when a jump gadget is
    encountered, the size of its corresponding byte string is guessed and a
    placeholder of the appropriate size that notes the target offset is
    generated. After which, following statements are compiled as above.

    After each statement has been translated into either bytes or place holders,
    the second phase translates the place holders into appropriate byte strings.
    If there are no corresponding bytes for the gadget in the library, then the
    search backtracks potentially up to the translation of the problematic jump
    statement in the first phase and guesses a different size. The backtracking
    process also considers different statement to gadget and gadget to byte
    translations on the way back to re-translating the problematic jump. This
    allows for the potential the target byte offset to changing without having
    to throw away all of the translations after the problematic jump.

    \section{Scaffolding}

    The code generation algorithm given above handles turning statements into
    bytes, but it does not handle linking or any of the other requirements of
    generating actual executables.

%    \begin{figure}
%        \begin{vcode}
%    .file "filename.igor"
%    
%    .LC_method:
%        .text
%        .globl    _method
%    _method:
%    .LFB_method:
%        push    %ebp
%        push    %edi
%        push    %esi
%        push    %ebx
%        mov    %esp, %ebp
%        sub    localVarSize, %esp
%        .byte  bytesForBody
%        add    localVarSize, %esp
%        pop    %ebx
%        pop    %esi
%        pop    %edi
%        pop    %ebp
%        ret
%    .LFE_method:
%        \end{vcode}
%        \caption[Example of hand written scaffolding.]{Example of hand written scaffolding for bootstrapping a method
%        that has been compiled to byte code.}
%        \label{tab:method-statements}
%    \end{figure}

    As this is a proof of concept system and not intended for production use,
    this implementation uses handwritten assembly code and tools from the GNU
    Compiler Collection to create object files for each method, and link them to
    form actual executables \cite{gcc}.

    This approach is ideal as it allows for easy inspection of the generated
    code in the object files, along with providing a mechanism for linking
    generated methods against C code to facilitate unit testing.

