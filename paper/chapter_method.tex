\chapter{Method}

    This section describes the theoretical underpinnings and the implementation
    of a gadget-based metamorphic engine. The engine works by maintaining a
    large library that maps from specific effects on execution state to bytes
    that will enact that transition when executed. Using this library and a
    semantic blueprint of a desired program, the engine generates an executable
    by nondeterministically piecing together bytes from the library in a way
    that fulfils the semantics of the program.

    \section{Gadgets}
    
    The idea of stitching together programs out of existing building blocks
    originated in the field of return-oriented programming \cite{rop_geo}.
    Return-oriented programming chains together small snippets of instructions
    ending in return statements through stack manipulations. Each of these
    snippets of instructions are called gadgets in the return-oriented
    programming community and each gadget typically performs a simple operation.

    The traditional formalization and use of gadgets has been heavily influenced
    by the constraints of return-oriented programming. Many of these imposed
    constraints are not present in the context of a metamorphic engine. For
    instance, the requirement that a gadget end in a return statement is
    necessary in return-oriented programming because that is the mechanism
    through which gadgets can be composed. However, the metamorphic engine is
    directly generating the text section of an executable this requirement is no
    longer necessary.

    Therefore, it is useful to re-formalize gadgets in the context where they
    are free of the constraints imposed by return-oriented programming. For the
    purposes of this engine, gadgets are defined by a pair of preconditions that
    described the execution state before performing the gadget, and a
    postconditions which describe the states after the gadget has been executed.

    To simplify the process of reasoning about a large collection of gadgets, a
    given bytestring is restricted to matching several of a collection of
    predefined gadget types. A Turing complete set of gadget types is given in
    \cite{franken}, and serves as the basis for the gadget types in this
    implementation. Each type of gadget is parameterized over a list of hardware
    locations\footnote{Memory addresses and registers}. A complete listing of
    gadgets and their definitions is given in table~\ref{tab:method-gadgets}.

    In table~\ref{tab:method-gadgets}, the set $C$ contains the set of hardware
    locations that the gadget overwrites in addition to performing it's intended
    semantics. Allowing gadgets to have limited side-effects describable by this
    ``clobber set'' allows for the inclusion of junk instructions in the
    bytestrings that map to any given gadget type. This is a desirable feature
    for a metamorphic engine because it leads to greater variety in the byte
    instances for each gadget type and through that more variety in the
    generated programs.  

    %Table of gadgets

    \section{Semantic Blueprints}

    Gadgets as they are used in this engine occupy a similar position in the
    hierarchy of semantic complexity as assembly level instructions. They are
    parameterized over hardware locations, which would require a programmer
    writing a program in gadgets to manually maintain such things as the stack
    frame and the mapping of registers and memory locations to variables and
    function arguments.

    For this reason it would be convenient to be able to express program
    semantics using a higher level construct. The solution used in this
    implementation is ``abstract statements''. Abstract statements, or simply
    statements, exist at a similar level of complexity, but abstract away the
    hardware locations into variables. This divorces the program semantics from
    the specific hardware that it will be running on, and opens the door for
    cross-platform compilation.

    %Table of statements

    \section{Gadget Discovery}

    The code generation algorithm presupposes a large library that maps from
    gadget parameterizations to bytestrings that implement them.

    \subsection{Byte Source}

    Before we can begin matching bytestrings to gadgets and inserting them into
    the library we must first decide where the bytes are going to come from.
    The method proposed by \cite{franken} is to scan the host computer and pull
    bytes from found executables. This approach has the desirable property that
    the bytes are being sampled from benign programs. There is a fatal problem
    with this approach however, and that is approach is that it requires
    interaction with the host environment in a way that is easily recognizable
    by behavioural detection systems due to it's uncommonly large number of disk
    seeks.

    A naive alternative to this approach would be to randomly generate bytes
    from a uniform distribution. This solution has several beneficial
    properties, namely there is no longer any interaction with the host
    environment at all, there is no longer a limited number of bytes that can be
    sampled. Along with these potential niceties there is a significant downside
    as well. The distribution that the bytes are being sampled from no longer
    has any connection to the distribution of bytes found in benign programs.

    A natural alteration to the naive approach is to pre-calculate the
    distribution of bytes found in benign programs and randomly sample bytes
    from this calculated distribution. This approach maintains all of the
    desirable benefits of the naive approach while avoiding the shortcomings of
    both the naive approach and the approach proposed in \cite{franken}. In
    addition, there is also the additional benefit that since the distribution
    is calculated in advance, it is no longer tied to the host architecture,
    which opens the door for cross-platform code generation.

    \subsection{Symbolic Evaluation}

    In order to match a given bytestring with the various gadget types that it
    may implement it is necessary to be able to represent the effect on hardware
    that the execution of the bytestring will have. Modelling the effect on
    hardware is done by dissassembling the bytestring and then sequentially
    symbolically evaluating the instruction. The result is a sequence of state
    forests that represent the progressive effects on hardware.

    Each state forest contains a tree for each affected hardware location.
    The tree in the forest is an expression tree where nodes are algebraic
    operations and leaves are the initial values of hardware locations.
    
    %Give a diagram of what the state trees can look like

                    Symbolically evaluate short sequences of bytes

                        Explain the symbolic representation

                        Aborts on non-deterministic/ambiguous machine states

                    Match gadgets with machine state after evaluating each
                    instruction in sequence

                        Disallow unsafe machine states

                        States that read or write to memory locations, i.e.
     
    \section{Code Generation}



            Code Generation

                High level overview

                    Resolve variables to registers and memory locations

                    Ensures nothing of value is clobbered during execution

                Algorithm

                    Pruning DFS

                    Maintain a set of clobberable/free registers

                    For each abstract statement

                        Non-Jumps

                            Load memory locations into registers

                                Not strictly necessary, but better odds of
                                finding a gadget parameterized over registers
                                than memory locations

                                Done using non-clobbering LoadMemReg

                                Allocate a temporary register from the free pool
                                for each value

                            Load constants into registers

                                Filter LoadConst gadgets that load a constant
                                that can be shifted right to give the desired
                                constant

                                If necessary allocate a temporary register
                                result

                                Find gadget that corresponds to the statement
                                with the given parameterization that only
                                clobbers the free registers

                                If required, store result into memory with
                                StoreMemReg

                        Jumps

                            Instances of the same gadget may be different sizes,
                            cannot know the byte offset of gadgets that come
                            later

                            Size of the gadget is guessed

                            Placeholder of guessed size containing all the
                            jump’s information is placed in the byte stream

                            As a second pass, each placeholder is resolved to a
                            gadget

                                if none is found, backtrack up the search tree


                    Scaffolding

                        Generated bytes are inserted between prewritten function
                        intro and outro

                            Not strictly necessary, but simplifies evaluation

                        Uses GCC/AS to compile the resulting bytes into an .o
                        file that can be linked against c programs for testing


%    \begin{table}
%        \centering
%        \begin{tabular}{|l|l|l|l|l|}
%            \hline
%            Gadget Type & Input & Output & Preconditions & Postconditions  \\
%            \hline
%            NoOp & &
%                &   
%                    \parbox[c][1.2cm]{3cm}{ }
%                &
%                    \parbox[c][1.2cm]{3cm}{
%                        \begin{align*}
%                            \forall c \in C. c & \leftarrow \phi
%                        \end{align*}
%                    }
%                \\
%            \hline
%            LoadReg & $L1$ & $L2$
%                &   
%                    \parbox[c][1.2cm]{3cm}{
%                        \begin{align*}
%                            a & \leftarrow L2 
%                        \end{align*}
%                    }
%                &
%                    \parbox[c][1.2cm]{3cm}{
%                        \begin{align*}
%                            L1 & \leftarrow a \\
%                            \forall c \in C. c & \leftarrow \phi
%                        \end{align*}
%                    }
%                \\
%            \hline
%            Plus & $L1$ & $L2,L3$
%                &   
%                    \parbox[c][1.2cm]{3cm}{
%                        \begin{align*}
%                            a & \leftarrow L2 \\ 
%                            b & \leftarrow L3
%                        \end{align*}
%                    }
%                &
%                    \parbox[c][1.2cm]{3cm}{
%                        \begin{align*}
%                            L1 & \leftarrow a + b \\
%                            \forall c \in C. c & \leftarrow \phi
%                        \end{align*}
%                    }
%                \\
%            \hline
%            Minus & $L1$ & $L2,L3$
%                &   
%                    \parbox[c][1.2cm]{3cm}{
%                        \begin{align*}
%                            a & \leftarrow L2 \\ 
%                            b & \leftarrow L3
%                        \end{align*}
%                    }
%                &
%                    \parbox[c][1.2cm]{3cm}{
%                        \begin{align*}
%                            L1 & \leftarrow a - b \\
%                            \forall c \in C. c & \leftarrow \phi
%                        \end{align*}
%                    }
%                \\
%            \hline
%ata Gadget = NoOp
%            | LoadReg       X.Register  X.Register
%            | LoadConst     X.Register  X.Value
%            | LoadMemReg    X.Register  X.Address
%            | StoreMemReg   X.Address   X.Register
%            | Plus          X.Register  (S.Set X.Register)
%            | Xor           X.Register  (S.Set X.Register)
%            | Times         X.Register  (S.Set X.Register)
%            | Minus         X.Register  X.Register          X.Register
%            | RightShift    X.Register  Integer -- arithmetic shift
%            | Compare       X.Register  X.Register
%            | Jump          Integer     X.Reason Integer
%        \end{tabular}
%        \caption{Enumeration of gadgets and their semantic definitions.}
%        \label{tab:method-gadgets}
%    \end{table}
