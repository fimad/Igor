\chapter{Related Work}

    Malware classification and more generally program analysis fall under two
    umbrella categories: static and dynamic analysis. Static methods analyze a
    program using only the byte level representation as information. Being able
    to statically analyze a program is obviously desirable in the context of
    malware analysis as it allows for the classification of malware before it
    performs malicious actions on the host.
    
    Modern antivirus software relies heavily on static analysis to classify
    malware. The current approach is to maintain a database of byte signatures
    (complex patterns of byte sequences) for every piece of malware ever
    encountered. If a program under investigation matches any of the byte
    signatures in the database then the program is classified as malicious.
    This technique is nice because it can be done relatively quickly which means
    that there is only minor disruption in the users work flow.

    Dynamic analysis gathers information about a program as it executes. When
    dealing with potentially malicious programs, as in the case of malware
    analysis, it is common practice is to run it in a sandboxed or emulated
    environment. Dynamic analysis covers techniques that include computing a
    live trace of instructions a program executes and anomaly detection which
    monitors a program interaction with the host for behaviour that deviates
    from normal program behaviour for some definition of normal. Sequences of
    syscalls or the api calls made over the lifetime of a program has been shown
    to be a good model of program behaviour \cite{api_calls}.

    Due to modern commercial anti-virus software's reliance on static analysis,
    a large portion of the evasion efforts staged by malware authors have been
    in the attempt to thwart byte signature detection by creating malware that
    obfuscates its byte signature for each host it infects.

    The earliest class of malware capable of automatically altering its byte
    signature is polymorphic malware. The defining attribute of polymorphic
    malware is that it uses encryption to randomize the appearance of the byte
    level representation of the program \cite{simile}. The simplest case of
    polymorphism partitions the malware into two segments, a small decryption
    routine and the payload. The payload is encrypted with a randomly chosen
    encryption key and upon execution the small decryption routine decrypts the
    body in memory and then transfers control to the main portion of the
    program. New generations of the program are created by altering the
    encryption key and applying minor syntactic variations to the decryption
    routine. The result is that the new binary has a widely different byte
    signature than it's predecessor. More sophisticated approaches to
    polymorphism include modularized encryption schemes where portions of the
    malware body are decrypted on the fly as they are needed and then
    re-encrypted.

    This form of obfuscation has largely been defeated through the use of hybrid
    detection schemes that use aspects of both dynamic and static analysis
    \cite{polyunpack}. In such schemes, the suspected malware is allowed to
    execute in a sandboxed environment while the anti-virus program monitors
    execution for behaviour characteristic of a packed program, e.g. passing
    control to a section of memory marked as data. The monitoring program can
    then dump the portion of memory that was jumped into, often times revealing
    the entire unpacked contents of the malware. Once the unpacked contents of
    the malware has been revealed it is then vulnerable to common byte signature
    detection.

    The next generation of obfuscating malware, metamorphic malware, achieves
    its obfuscation by applying semantics-preserving transformations to the
    bytecode representation of its instructions. The result is a program that
    executes a unique sequence of instructions while performing the same
    semantic operation. There are several existing metamorphic engines of
    varying complexity though most implement similar transformations which
    include: movement and interleaving of logical sequences of instructions,
    insertion of random non-executed instructions, and swapping of registers
    \cite{simile}. 

    Theoretically, metamorphic obfuscation is very strong. Reliably identifying
    a piece of malware that has undergone metamorphic transformations has been
    shown to be NP-complete \cite{npcomplete}. Further, there are known methods
    of control flow obfuscation that make simply determining the control flow
    graph of a program NP-hard \cite{controlflow}. These proofs pertain to
    static analysis and provides a bound on the ability to statically classify
    metamorphic malware. In practice however, there are static techniques being
    researched that have been shown to reliable classify modern metamorphic
    malware.

    The key observation that makes classifying current metamorphic malware
    tractable is that while the sequences of bytes exhibited between various
    generations of metamorphic malware differs significantly, as a whole they
    exhibit certain statistical characteristics that allow them to be
    identified. For instance, the opcode frequency distributions of metamorphic
    variants has been shown to be relatively constant between variations. Opcode
    histogram alone have been shown to be robust enough in some cases to
    reliably classify real world metamorphic malware samples \cite{histogram}.

    Another significant property exhibited by modern malware engines is that not
    only do the executables they generate have certain statistical properties in
    common, but they are also statistically very dissimilar from benign software
    \cite{hmm_detect}. HMM-based classifiers that capitalize on this observation
    have been shown to be particularly successful in classifying metamorphic
    malware.

    Given this shortcoming with regard to statistical fingerprints in current
    metamorphic engines, recent work has explored creating metamorphic engines
    that are able to statistically blend in with benign applications. In some
    instances it has been shown that simply embedding whole subroutines taken
    from benign applications randomly through out a malware sample is enough to
    fool certain classifiers \cite{hmm_evade}.

    More sophisticated classification models that take into account the opcode
    instruction frequencies for the entire executable are more robust to these
    sorts of evasion tricks. Chi-squared tests have been shown to be more
    sensitive to the frequency histograms and thus are more robust to the minor
    alterations that occur when embedding portions of benign programs
    \cite{chisquared}.

    More recent advances in obfuscation have focused on altering the entire
    frequency histogram of a metamorphic variation to more closely resemble
    those of benign software. One particular system, which the authors referred
    to as mimimorphism, employs mimic functions to generate each variation
    \cite{mimimorph}. The mimic functions create Huffman forests to represent
    the frequency distribution of opcodes. Given the initial version of a
    program and a pseudo-random number generator, the mimimorphic engine walks
    the Huffman forest to generate a new obfuscated version.

    Other techniques build on the idea of incorporating benign code in each
    generation. Instead of simply using benign code fragments as dead code,
    these engines scan benign executables to generate a database of repurposable
    code fragments from which subsequent generations of the program are pieced
    together \cite{franken}. The intellectual roots for this type of metamorphic
    engine comes from a recently discovered shellcode execution attack known as
    return-oriented programming.

    Classic examples of buffer overflow attacks involve the attacker overflowing
    a buffer allocated on the stack filling it with instructions that will spawn
    a shell \cite{aleph}. The portion of the content copied into the buffer that
    overwrites the return address of the stack frame contains an address that
    points into the overrun buffer. The result is that when the current function
    exits by executing the ``ret'' instruction, control will pass to the
    injected code thus giving the attacker the ability to execute arbitrary
    code.

    Since the publication of \cite{aleph} there have been several advances that
    make classic buffer overflows much more difficult to execute. One of the
    most notable examples of these measures is the introduction of W$\oplus$X
    memory which marks every segment of memory allocated to a process as either
    writable or executable but not both. The result of this is that anywhere the
    attacker may be able to inject malicious code will have permissions set that
    disallow it from being run.

    These measures spurred a new form of attack known as ``return to libc''. In
    this scenario, the attacker capitalizes on the fact that most programs are
    linked to libc, which contains functions that can be used to perform
    potentially malicious operations (notably the ``system'' method which
    executes a string argument in a shell). When the attacker overruns a buffer
    they are able to control the contents of the stack frame. In \emph{x86}
    architectures, the return address and function parameters are passed on the
    stack. By specially crafting the contents of the stack, the attacker can set
    the return address to the libc ``system'' method and place a malicious
    command as the parameter thereby executing an arbitrary shell command when
    the current function exits.

    Return oriented programming builds on the key idea seen in return to libc
    attacks of reusing existing instructions in a malicious manner. Return
    oriented programming generalizes the idea by noting that because the return
    address of a method is stored on the stack, and that the attacker can
    control the contents of the stack, they can chain together code fragments
    (gadgets) from the victim executable that end in a ``ret'' opcode. As each
    code fragment ``returns'' the address of the next fragment is obtained from
    the tainted stack frame. It has been shown that the libc library alone
    contains sufficient and varied gadgets to constitute a Turing complete
    system \cite{rop_geo}.

    The ``Frankenstein'' metamorphic engine in \cite{franken} combines the
    principles of return oriented programming with the observation that programs
    statistically similar to benign programs are more difficult for static
    analysis techniques to classify. The presented metamorphic systems builds a
    collection of gadgets by scanning through a sampling of benign programs
    found on the host. There are several simplifying assumptions that can be
    made in the case of the metamorphic engine that cannot be made in standard
    return-oriented programming. The most significant is that gadgets do not
    need to return in a ``ret'' instruction for them to be useful. Given the
    collection of discovered gadgets, metamorphic programs are written as
    predicates which are then mapped to gadgets according to the gadgets
    description of its effect on the system.

