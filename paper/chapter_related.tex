\chapter{Related Work}

    Malware classification techniques fall under two categories: static and
    dynamic analysis. As the name suggests, static methods analyze the byte
    structure of a given sample in an attempt to classify malware without
    running it. Contrarily, dynamic analysis allows the sample to run (typically
    in a sandboxed or emulated environment) and gathers information about the
    sample as it executes. Based on the observations made during execution, the
    sample is classified as benign or malicious. Anomaly based detection
    systems are a special class of dynamic analysis that watch for program
    behaviour that deviates from some predefined normal program behaviour.
    Sequences of syscalls or the api calls made over the lifetime of a program
    has been shown to be a good model of program behaviour \cite{api_calls}.

    Contrary to dynamic analysis, static analysis attempts to classify a sample
    before it is permitted to run. Typically this involves matching a potential
    malware sample against byte signatures (complex patterns of byte sequences)
    from a database of known malware.  Modern commercial anti-virus vendors
    largely rely on static analysis for classifying malware. For this reason, a
    large portion of the evasion efforts staged by malware authors have been in
    the attempt to thwart static analysis by creating malware that has the
    capability of modifying its byte signature.

    The earliest class of malware capable of the automated altering its byte
    signature is known as polymorphic malware, and employs encryption schemes to
    randomize the appearance of the malware \cite{simile}.  The simplest case of
    polymorphism partitions the malware into two segments, a small decryption
    routine and the main body. The main body is encrypted with a randomly chosen
    encryption key and upon execution the small decryption routine decrypts the
    body in memory and then transfers control to the main portion of the
    program.  New generations of the program are created by altering the
    encryption key and applying minor syntactic variations to the decryption
    routine. The resulting binary has a widely different byte signature than
    it's predecessor.  More sophisticated approaches to polymorphism include
    modularized encryption schemes where portions of the malware body are
    decrypted on the fly as they are needed and then re-encrypted.

    This form of obfuscation has largely been defeated through the use of hybrid
    detection schemes that use aspects of both dynamic and static analysis
    \cite{polyunpack}.  In such schemes, the suspected malware is allowed to
    execute in a sandboxed environment, while the anti-virus program monitors
    execution for behaviour characteristic of a packed program, e.g. passing
    control to a section of memory marked as data. The monitoring program can
    then dump the portion of memory that was jumped into, often times revealing
    the entire unpacked contents of the malware.

    The next generation of obfuscating malware, metamorphic malware, achieves
    obfuscation by applying semantics-preserving transformations to the bytecode
    representation of its instructions. The result is a program that executes a
    unique sequences of instructions while performing the same operation. There
    are several existing metamorphic engines of varying complexity, though most
    implement similar transformations including: the movement and interleaving
    of logical sequences of instructions, the insertion of random non-executed
    instructions, and the swapping of registers \cite{simile}. 

    Theoretically, metamorphic obfuscation is very strong. Reliably identifying
    a piece of malware that has undergone metamorphic transformations has been
    shown to be NP-hard \cite{npcomplete}. Further, there are known methods of
    control flow obfuscation that make simply determining the control flow graph
    of a program NP-Hard \cite{controlflow}.  This may seem damning for
    anti-virus vendors, fortunately however, these proofs only apply to static
    methods of classification and in practice current metamorphic malware has
    not reached sophisticated enough levels to achieve such difficulty in
    classification.

    The key observation that makes classifying current metamorphic malware
    tractable is that while the sequences of bytes exhibited between various
    generations of metamorphic malware differs significantly, current
    generations of metamorphic malware exhibit certain statistical
    characteristics that enable them to be classified. For instance, the opcode
    frequency distributions of metamorphic variants has been shown to be
    relatively constant between variations. Opcode histogram alone have been
    shown to be robust enough in some cases to reliably classify real world
    metamorphic malware samples \cite{histogram}.

    Another significant property exhibited by modern malware engines is that not
    only do the executables they generate have statistical properties in common,
    but they are also statistically very dissimilar from normal benign software
    \cite{hmm_detect}. HMM-based classifiers that capitalize on this observation
    are particularly accurate in classifying metamorphic malware.

    Given this shortcoming with regard to statistical fingerprints in current
    metamorphic engines, recent work has explored creating metamorphic engines
    that are able to statistically blend in with benign applications.  In some
    instances it has been shown that simply embedding whole subroutines taken
    from benign applications randomly through out a malware sample is enough to
    fool certain classifiers \cite{hmm_evade}.

    More sophisticated classification models that take into account the opcode
    instruction frequencies for the entire executable are more robust to these
    sorts of evasion tricks. Chi-squared tests have been shown to be able to
    more sensitive to the frequency histograms and thus are more robust to the
    minor alterations that occur when embedding portions of benign programs
    \cite{chisquared}.

    More recent advances in obfuscation have focused on altering the entire
    frequency histogram of a metamorphic variation to more closely resemble
    those of benign software. One particular system the authors referred to as
    mimimorphism employs mimic functions to generate each variation
    \cite{mimimorph}. The mimic functions used create Huffman forests to
    represent the frequency distribution of opcodes, and then given the initial
    variation and a pseudo-random number generator walk the Huffman forest to
    generate a new version.

    Other techniques build on the technique of incorporating benign code in each
    generation. Instead of simply using benign code fragments as dead code,
    these engines scan benign executables to generate a database of repurposable
    code fragments from which subsequent generations of the program are pieced
    together \cite{franken}. The intellectual roots for this type of metamorphic
    engine comes from a recently discovered shellcode execution attack known as
    return-oriented programming.

    Classic examples of buffer overflow attacks involve the attacker overflowing
    a buffer allocated on the stack filling it with instructions that will spawn
    a shell \cite{aleph}. The portion of the content copied into the buffer that
    overwrites the return address of the stack frame contains an address that
    points into the overrun buffer. The result is that when the current function
    exits by executing the ``ret'' instruction, control will pass to the
    injected code thus giving the attacker the ability to execute arbitrary
    code.

    Since the publication of \cite{aleph} there have been several advances that
    make classic buffer overflows much more difficult to execute. One of the
    most notable examples of these measures is the introduction of W$\otimes$X
    memory which marks every segment of memory allocated to a process as either
    writable or executable but not both. The result of this is that anywhere the
    attacker may be able to inject malicious code will have permissions set that
    disallow it from being run.

    These measures spurred a new form of attack known as ``return to libc''. In
    return to libc attacks, the attacker capitalizes on the fact that most
    programs are linked to libc, which contains functions that can be used to
    perform potentially malicious operations (notably the ``system'' method
    which executes the passed in string in a shell). When the attacker overruns
    a buffer the are able to control the contents of the stack frame.  In x86
    architectures, the return address and function parameters are passed on the
    stack. By specially crafting the contents of the stack, the attacker can set
    the return address to the libc ``system'' method and place a malicious
    command as the parameter thereby executing an arbitrary shell command when
    the current function exits.

    Return oriented programming builds on the key idea seen in return to libc
    attacks of reusing existing instructions in a malicious manner. Return
    oriented programming generalizes the idea by noting that because the return
    address of a method is stored on the stack, and that the attacker can
    control the contents of the stack, they can chain together code fragments
    (gadgets) from the victim executable that end in a ``ret'' opcode. As each
    code fragment ``returns'' the address of the next fragment is obtained from
    the tainted stack frame.  It has been shown that the libc library alone
    contains sufficient and varied gadgets to constitute a Turing complete
    system \cite{rop_geo}.

    The metamorphic engine in \cite{franken} combines the principles of return
    oriented programming with the observation that programs statistically
    similar to benign programs are more difficult for static analysis techniques
    to classify. The described metamorphic systems builds a collection of
    gadgets by scanning through a sampling of benign programs found on the host.
    There are several simplifying assumptions that can be made in the case of
    the metamorphic engine that cannot be made in standard return-oriented
    programming. The most significant is that gadgets do not need to return in a
    ``ret'' instruction for them to be useful. Given the collection of
    discovered gadgets, metamorphic programs are written as predicates which are
    then mapped to gadgets according to the gadgets description of its effect on
    the system.

