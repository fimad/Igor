\documentclass[finalcopy,short]{srpaper}
\title{Extended Abstract}
\author{Will Coster}
\date{November 20, 2012}

\begin{document}
  \frontmatter

  Existing metamorphic engines while producing variations that on the surface
  differ widely among each other do a poor job of producing variations that are
  statistically similar to benign programs. The metamorphic engine presented in
  \cite{franken} presents one possible solution to this whereby variations are
  pieced together from code fragments taken from benign programs on the host
  system. However, a pitfall of this approach is that while it may be statically
  difficult to detect, dynamically it's behaviour is very suspicious. Also, due
  to its reliance on gadgets, instructions with the desired constants,
  especially if they are 32bit words, maybe rare or nonexistent in the library
  of gadgets.

  We present a behaviourally benign method of generating variations of a program
  that are both statistically similar to benign programs and also exhibit a high
  degree of variation between generations. The variation between generated
  instances is achieved by nondeterministically piecing together instances from
  a library of gadgets. Similarity to benign programs is achieved by drawing
  gadgets from a random byte stream conforming to the observed byte frequencies
  of benign programs. If the byte histograms are encoded at the initial
  generation, then the system requires no interaction with the host system to
  generate subsequent instances. Further, we also present a method for resolving
  predicates which require rare constants by combining algebraic gadgets to
  generate the desired constants or memory locations. 

  The resulting system will be evaluated on the achievable variation between
  generated instances and on the overall similarity to benign programs.
  Evaluation of the variation between instances can be measured against the
  results of \cite{franken} and other existing metamorphic engines. Evaluation
  of the similarity to benign programs will be achieved by measuring the
  similarity of instances generated via various methods of modelling the byte
  frequencies. Baseline performance will be measured against instances generated
  from a random byte stream with a uniform distribution of byte frequencies.

  \bibliography{annbib} % Insert the name of your bibliography here.
\end{document}

