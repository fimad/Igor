\documentclass[finalcopy,short]{srpaper}

\title{Literature Review}
\author{Will Coster}
\date{November 20, 2012}

\begin{document}
  \frontmatter
  \nocite{*}

  Modern commercial anti-virus software largely relies on static analysis for
  classifying malware. In response to this malware authors have attempted to
  write programs that are able to modify their byte signatures with every
  execution.
  
  The fist method of altering a programs byte signature is through encryption
  \cite{simile}.  The main body of the malware is encrypted and the entry point
  is defined to be a small decryption routine that decrypts the body and then
  transfers control to the main portion of the virus. By changing the encryption
  key and applying minor syntactic variations to the decryption routine, the
  resulting binary has a widely different byte signature to it's predecessor.

  More sophisticated approaches to polymorphism include modularized encryption
  where portions of the malware body are decrypted as they are needed and then
  re-encrypted.

  This form of obfuscation has largely been defeated through the use of hybrid
  detection schemes that use both dynamic and static analysis \cite{polyunpack}.
  In such schemes, the suspected malware is allowed to execute in a sandboxed
  environment, and the anti-virus program then monitors execution for behaviour
  characteristic of a packed program, e.g. passing control to a section of
  memory marked as data.

  The following generation of malware achieves obfuscation by applying
  semantic-preserving transformations to it's bytecode instructions. The result
  is a program consisting of a different sequences of instructions but performs
  the same operations. The most common techniques employed in metamorphic
  malware are code permutation, the insertion dead code, and register swapping.

  ~\\

  On the theoretic side, metamorphic malware is very strong. Reliably
  classifying as equal two pieces of malware that have undergone metamorphic
  transformations has been shown to be NP-hard \cite{npcomplete}. In practice
  however, there are methods of detection that have been shown to work well for
  the classes of metamorphic malware that is commonly seen in the wild.

  Despite the ability \cite{hmm_detect}


  ~\\

  History of malware
  polymorphic
  metamorphic

The idea of blending in with the benign
  mimimorph
  frankenstein

critique of frankenestien that while it may be constructed from benign programs,
the signatures anti virus vendors use are long regex like and span many bytes.
If a signature matches bytes that correspond to one instruction per-gadget, the
added noise doesn't really supply any benefit.



  \bibliography{annbib} % Insert the name of your bibliography here.
\end{document}
