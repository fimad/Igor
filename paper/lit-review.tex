\documentclass[finalcopy,short]{srpaper}

\title{Literature Review}
\author{Will Coster}
\date{November 20, 2012}

\begin{document}
  \frontmatter
  \nocite{*}

  Modern commercial anti-virus software largely relies on static analysis for
  classifying malware. In response to this malware authors have attempted to
  write programs that are able to modify their byte signatures with every
  execution.
  
  The fist method of altering a programs byte signature is through encryption
  \cite{simile}.  The main body of the malware is encrypted and the entry point
  is defined to be a small decryption routine that decrypts the body and then
  transfers control to the main portion of the virus. By changing the encryption
  key and applying minor syntactic variations to the decryption routine, the
  resulting binary has a widely different byte signature to it's predecessor.

  More sophisticated approaches to polymorphism include modularized encryption
  where portions of the malware body are decrypted as they are needed and then
  re-encrypted.

  This form of obfuscation has largely been defeated through the use of hybrid
  detection schemes that use both dynamic and static analysis \cite{polyunpack}.
  In such schemes, the suspected malware is allowed to execute in a sandboxed
  environment, and the anti-virus program then monitors execution for behaviour
  characteristic of a packed program, e.g. passing control to a section of
  memory marked as data.

  The following generation of malware achieves obfuscation by applying
  semantic-preserving transformations to it's bytecode representation. The
  result is a program consisting of a different sequences of instructions but
  performs the same operations. The most common techniques employed in
  metamorphic malware are code permutation, the insertion dead code, and
  register swapping \cite{simile}.

  Theoretically, metamorphic obfuscation is very strong. Reliably identifying a
  piece of malware that has undergone metamorphic transformations has been shown
  to be NP-hard \cite{npcomplete}. This may seem damning for anti-virus vendors,
  fortunately however the proof only applies to static methods of classification
  and in practice current metamorphic malware is typically not sophisticated
  enough to achieve such difficulty in classification.

  \cite{histogram} found that while the sequences of bytes exhibited between
  various generations of metamorphic malware differs significantly, the
  frequency of opcodes is relatively static. Using a classifier based on opcode
  histograms, \cite{histogram} was able to reliably detect many examples of
  real world malware.

  \cite{hmm_detect} presents a method of static analysis that at the time was
  able to reliably classify malware that had been obfuscated with popular
  metamorphic engines. The classifier presented in the paper relied on the
  observation that while existing metamorphic engines are able to achieve a high
  degree of dissimilarity between variations, they also exhibit a high degree of
  dissimilarity from benign programs.

  Given the trend in classification to rely on the dissimilarity between
  metamorphic malware and benign programs, recent work has explored creating
  metamorphic engines that are able to statistically blend in with benign
  applications.  \cite{hmm_evade} suggests that simply embedding subroutines
  from benign applications in the malware is enough to fool classifiers similar
  to those presented in \cite{hmm_detect}.

  \cite{mimimorph} presents a method of obfuscation they call mimimorphism that
  generates code that has similar statistical properties to a given population
  of files. The technique uses Huffman encoding .....


  ~\\

  History of malware
  polymorphic
  metamorphic

The idea of blending in with the benign
  mimimorph
  frankenstein

critique of frankenestien that while it may be constructed from benign programs,
the signatures anti virus vendors use are long regex like and span many bytes.
If a signature matches bytes that correspond to one instruction per-gadget, the
added noise doesn't really supply any benefit.



  \bibliography{annbib} % Insert the name of your bibliography here.
\end{document}
